% \documentclass[12pt,notes,hyperref={unicode},aspectratio=169]{beamer}
\documentclass[12pt,presentation,hyperref={unicode},aspectratio=169]{beamer}
%\includeonlyframes{test}
\usepackage[utf8]{inputenc}

\usepackage[english]{babel}
\usepackage{forloop}
\usepackage{multirow}

\usepackage{amsmath,hyperref,multimedia, bbold, amssymb}

\usepackage{minted}
\usepackage{wrapfig, multicol}

% \AtBeginSection[]
% {
%   \begin{frame}<beamer>{Outline}
%     \tableofcontents[currentsection,currentsubsection]
%   \end{frame}
% }

\mode<presentation>
\usetheme{default}
\usecolortheme{beaver}
\setbeamertemplate{navigation symbols}{
  {\small
    \insertframenumber/\inserttotalframenumber
  }
}

\subtitle{Meeting C++ 2018}
\title{Competitive STL Extensions}
\author{Fedor Alekseev}
\institute{Moscow Institute of Physics and Technology: My pity}
\date{\today}

\newmintedfile[cppfile]{c++}{linenos}
\newmintedfile{asm}{linenos}

\begin{document}

\begin{frame}
  \titlepage
  \note{
    Good evening everyone!
    Thanks for having me.
    My name is Fedor.
    I'm a student from Moscow.

    I'm also doing some competitive programming as a hobby.
    I have a team called My pity.
  }
\end{frame}

\begin{frame}{Outline}
  \tableofcontents
  \note{
    This talk will be about some non-standard library tools that are available
    for g++ users and that are especially valuable for competitive programmers.
  }
\end{frame}

\section{Competitive Programming}

\begin{frame}{Competitive Programming}
  \note{
    So, during a programming contest you are usually given a set of
    well-defined algorithmic problems.
    You or your team are usually given 4 to 12 problems and 2 to 5 hours to solve them.
    For each problem, you have to come up with a correct solution, and then
    code it and submit the source code to the judging system.
  }

  \begin{block}{A contest}
    \begin{itemize}
      \item<1-> Participants receive a set of 4 to 12 problems, and they have 2
        to 5 hours to solve them.
      \item<2-> Problems are well-defined and usually have computer science
        nature
      \item<3-> Solving a problem means sending a program to the judging system.
        The program should pass all the secret test cases within time limit
      \item<4-> Optimal algorithmic complexity is usually enough, especially for
        C++ solutions
      \item<5-> Solutions are compiled in a judging environment without any
        additional libraries, with just a vanilla compiler installation.
    \end{itemize}
  \end{block}
\end{frame}

% \begin{frame}{Competitive Programming}
%   Engineering is Programming integrated over time?
% \end{frame}

\section{Kool tricks}

\subsection{Standard library}
\begin{frame}[fragile]{Standard library}
  \begin{itemize}
    \item<1-> Algorithms: \mintinline{c++}|sort, lower_bound, unique, next_permutation,|~etc
    \item<1-> Data structures: \mintinline{bash}|{unordered_,}{set,map}|,
      simpler containers
    \item<2-> GNU C++ specific:
    % \begin{itemize}
      % \item<3->
      %   \mintinline{c++}|#include <algorithm>| includes
      %   \mintinline{c++}|std::__gcd|, even in pre-C++17 mode
      % \item<4->
        \mintinline{c++}|#include <bits/stdc++.h>| includes everything!
    % \end{itemize}
  \end{itemize}
\end{frame}

\subsection{g++ builtins}

\begin{frame}[fragile]{popcount: number of set bits}
  \cppfile{popcnt.cc}
  godbolts under x86 to
  \asmfile{popcnt.asm}
  Similarly,
  \mintinline{c++}|__builtin_clz| and \mintinline{c++}|__builtin_ctz|
  count leading/trailing zeros
\end{frame}

\subsection{SGI STL extensions}

% \begin{frame}[fragile]{SGI STL extensions: power}
%   \begin{block}{ext/numeric}
%     \cppfile[firstline=70,lastline=74,autogobble]{/usr/include/c++/8.2.1/ext/numeric}
%   \end{block}
% \end{frame}

\begin{frame}{SGI STL extensions: power}
  \begin{itemize}
    \item<1-> Sometimes you have an operation $(a^n)$ that can be
      expressed as some other operation $(a \cdot a)$ repeated $n$ times
    \item<2-> This is usually called exponentiation
    \item<3-> Matrix exponentiation:
      $A^n = E
        \cdot \underbrace{A \cdot A \cdot \ldots \cdot A}_{n \text{ times}}$,
        where $E$ is identity matrix
    \item<4-> Integer exponentiation with modulo: $$
      a^n \mod{p} = 1
      \cdot \underbrace{(((a \mod{p}) \cdot a \mod{p}) \cdot
      \ldots \cdot a \mod{p})}_{n\text{ times }a}
    $$
    \item<5-> Can be done in just $O(\log{n})$ multiplications
  \end{itemize}
\end{frame}

\begin{frame}[fragile]{SGI STL extensions: power}
  \cppfile[mathescape]{power_modulo.cc}
\end{frame}

% \begin{frame}{SGI STL extensions: rope}
%   rope?
% \end{frame}

\subsection{Policy-Based Data Structures}

\begin{frame}[fragile]{Policy-Based Data Structures}
  \begin{itemize}
    \item<1-> There are data structures fundamentally similar to these in the
      standard library, but with different trade-offs and possibilities
    \item<2-> Policy-Based Data Structures library is an attempt to express
      some of this variety
    \item<3-> Shipped with GNU C++ library as an extension within namespace
      \mintinline{c++}|__gnu_pbds|
  \end{itemize}
\end{frame}

\begin{frame}[fragile]{PBDS: order statistics tree}
  \note{
    std::set is cool: it implements a dynamic sorted sequence.
    The problem is that, well, although the order of this sequence is
    deterministic, and you even have fast access to the elements in the middle
    of the sequence.
    That is, by-value access.
  }
  \cppfile[firstline=1,lastline=12]{order_statistics.cc}
\end{frame}

\section{Lacking utilities}

\begin{frame}{Lacking utilities}
  \begin{itemize}
    \item<1-> C++ is a great choice for competitive programming, but
    \item<2-> There are some lacking utilities that still tamper its dominance
    \item<3-> Most importantly, arbitrary precision arithmetics:
      although problems requiring it are quite rare, sometimes it is easier to
      switch to python or java just for big integers.
  \end{itemize}
\end{frame}

\begin{frame}{kthxbye}
  \begin{itemize}
    \item Thanks!
    \item More examples are available on my github
      \url{https://github.com/moskupols/competitive-stl-extensions}
    \item For more info on PBDS see GNU C++ library manual:
      \url{https://goo.gl/PmR86Z}
  \end{itemize}
\end{frame}

\end{document}
